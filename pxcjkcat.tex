% upLaTeX 文書; 文字コードは UTF-8
\documentclass[uplatex,dvipdfmx,a4paper]{jsarticle}
\usepackage{geometry}
\usepackage{xcolor}
\usepackage[colorlinks,hyperfootnotes=false]{hyperref}
\hypersetup{linkcolor=blue!75!black,urlcolor=green!45!black}
\usepackage{shortvrb}
\MakeShortVerb{\|}
\usepackage{verbatim}
\newenvironment{myverbatim}
  {\begin{quote}\small\verbatim}
  {\endverbatim\end{quote}}
\newcommand{\PkgVersion}{1.1-pre}
\newcommand{\PkgDate}{2012/09/22}
\newcommand{\Pkg}[1]{\textsf{#1}}
\newcommand{\Meta}[1]{$\langle$\mbox{}#1\mbox{}$\rangle$}
\newcommand{\Note}{\par\noindent ※}
\newcommand{\Means}{:\quad}
\newcommand{\jemph}{\textsf}
\newcommand{\wbr}{\linebreak[0]}
\providecommand{\pTeX}{p\TeX}
\providecommand{\upTeX}{u\pTeX}
\providecommand{\pLaTeX}{p\LaTeX}
\providecommand{\upLaTeX}{u\pLaTeX}
%-----------------------------------------------------------
\begin{document}
\title{\Pkg{pxchfon} パッケージ}
\author{八登崇之\ (Takayuki YATO; aka.~``ZR'')}
\date{v\PkgVersion\quad[\PkgDate]}
\maketitle

\begin{abstract}
本パッケージは、
{\upTeX}の和文文字カテゴリ(kcatcode)を扱う{\LaTeX}上の
インタフェースを提供する。
\end{abstract}

%===========================================================
\small
\begin{verbatim}
■ 本ソフトウェアの作者のサイト

    En toi Pythmeni tes TeXnopoleos ~電脳世界の奥底にて~
    http://zrbabbler.sp.land.to/

    ※ 以下のページに一部機能の使用例を紹介した。
    「upLaTeX を使おう」
    http://zrbabbler.sp.land.to/uplatex.html
    「PXbase パッケージ」
    http://zrbabbler.sp.land.to/pxbase.html

■ インストール

    TDS 1.1 に従ったシステムでは、各ファイルを次の場所に移動する。
    ・*.sty        → $TEXMF/tex/platex/PXcjkcat/
    (残りのファイルは不要)

    W32TeX を C:\usr\local にインストールした場合の例。
    ・*.sty      → C:\usr\local\share\texmf-local\tex\platex\PXcjkcat


-----------------------------------------------------
 pxcjkcat パッケージ (v1.0) -- 和文文字カテゴリ操作 
-----------------------------------------------------

■ 対応環境

    upLaTeX2e (v0.11 以降)

■ 読込

    \usepackage[<オプション>]{pxcjkcat}

    オプションとして以下のものが指定できる。

    ccv1
      「モード CCV」を 1 (upTeX v0.11~0.28 と互換)とする。
    ccv2
      「モード CCV」を 2 (upTeX v0.29 以降と互換)とする。
      ※「モード CCV」については後で詳述する。
    \cjkcategorymode で有効なモード値
      kcatcode がモードに従って設定される。これ以外の場合は、パッケージ
      読込時の kcatcode の変更はない。

■ 機能

    \cjkcategory{<ブロック>,...}{<カテゴリ>}
      <ブロック> で表される文字ブロック(複数指定が可能)の kcatcode を
      <カテゴリ> に変更する。<ブロック> は「ブロック ID (後述)」または
      「ASCII 以外の文字」で指定され、後者の場合はその文字の属するブロック
      を指す。<カテゴリ> は以下に示す様に「カテゴリ ID」または数値で指定
      する。この変更は局所的(グルーピングに従う)である。
      - noncjk (15): 欧文扱い
      - kanji [または han] (16): 漢字扱い
      - kana (17): 仮名扱い
      - cjk (18): 「その他の和文」扱い
      - hangul (19): ハングル扱い
      ブロック ID の一覧および各々のカテゴリの意味については後掲。

    \cjkcategorymode{<モード>}
      全てのブロックの kcatcode の一括設定を行う。モードには以下のものが
      ある。後に揚げるものほど noncjk であるブロックが増える。各モードで
      の具体的な設定値については「各モードでの kcatcode の値」の節を参照。
      - forcecjk : upTeX の既定の設定と同じ。(ただし「文字分類バージョン」
        による小さい差異がある。詳細については「文字分類バージョン」の節を
        参照。) ASCII 文字のみ noncjk で他のブロックは全て和文扱い(noncjk
        以外; 具体的な値は後述)。和文フォントの中の Unicode 値の割当がある
        全ての文字を和文文字として直接用いることができる。
      - prefercjk : 和文扱いのブロックとして、forcecjk のものに加えて、
        Adobe の定める CJK 文字集合(Adobe-Japan1, Adobe-GB1, Adobe-CNS1,
        Adobe-Korea1)の何れかと共通部分をもつ文字ブロックのみを加えて
        (具体的な値は forcecjk と同じ)、残りを noncjk に設定する。
      - prefercjkvar : prefercjk とほぼ同じで、違いは grek, grek1, cyrl
        の 3 つ(結果的にギリシャ・キリル文字の全て)が noncjk であること。
      - prefernoncjk : 以下に掲げる「必要最低限」のブロックを除き、全てを
        noncjk にする。
        * kanji: 漢字・部首・注音字母: hani, haniA, haniB, haniC, hani1,
            hani2, cjk01, cjk02, cjk03, cjk05, cjk06, bopo, bopo1.
        * kana: ひらがな・カタカナ: hira, kana, kana1.
        * cjk: CJK 記号の一部・全角/半角互換形・彝文字: cjk04, cjk08, 
            cjk07, cjk09, cjk10, cjk11, cjk12, cjk13, sym15, yiii, yiii1.
        * hangul: ハングル完成形・ハングル字母: hang, hang1, hang2, hangA,
            hangB.
        ※モード CCV が 2 の場合、cjk12 の再分割の中の cjk1b, cjk1c は
        kana に変更される。

■ kcatcode の値の意味

    upTeX (v0.11~) では Unicode 文字を複数の文字ブロックに分類し、各文字
    ブロック毎に kcatcode と呼ばれるパラメタを持たせている。(文字ブロック
    の分類は Unicode 文字ブロックとほぼ一致する。) これはそのブロックに
    属する文字が入力ソース中に現れた時の upTeX の字句解析の動作を規定する。
    なお以下では、入力文字コードが UTF-8 であると仮定する。また文字が属する
    ブロックの kcatcode のことを単にその文字の kcatcode と呼ぶ。

    (1) ある文字の kcatcode が 15 (noncjk) である場合、その文字は和文文字
      として扱われず、UTF-8 表現のバイト列とみなされる。字句解析の結果は
      それらのバイトの catcode の値に依存する。例えば、文字 α(U+03B1) の
      kcatcode が 15 の状態で α がソースに現れたら、欧文 TeX において
      ^^8e^^b1 というバイト列が現れた時と同じ結果になる。ここで inputenc
      で utf8 (または utf8x) を指定してあれば、inputenc における「文字 α」
      の処理に回るはずである。

    (2) ある文字の kcatcode が 15 以外である場合、その文字は 1 つの文字と
      して扱われ、次の catcode をもつ文字と同様に字句解析される。
      - kcatcode が 18 (cjk) → catcode 12 と同様
      - kcatcode が 16, 17, 19 → catcode 11 と同様
      そして、コントロールシーケンスの一部でないと判断された場合は、それ
      は指定された kcatcode を保持する和文文字トークンとなる。

    (3) kcatcode が 16, 17, 18 の和文文字トークンの直後の改行文字は無視
      される(pTeX の和文と同じ)。kcatcode が 19 の和文文字トークンの直後
      の改行文字は空白トークンとなる(欧文と同じ)。これはハングルかななる
      文書の組版に好都合だからである。

    補足:
      - ASCII 文字は常に欧文 TeX と同じ扱いになる。従って、ASCII ブロック
        (latn) の kcatcode 指定は意味をもたない。
      - kcatcode を表す upTeX のプリミティブは \kcatcode<数値> で、これは
        その数値を符号位置とする文字が属するブロックの kcatcode を指し示す
        レジスタを表す。

■ 文字分類バージョン

    upTeX での文字ブロックの分割(および各ブロックの kcatcode 値の既定値)は
    改版時に変更され、これが互換性の問題を起こす可能性がある。
      a. Unicode の改版での文字ブロックの追加に追随する為の変更。
      b. 複数の文字種が混在する 1 つの Unicode 文字ブロックに対し、その中
         で文字種により処理を変えたいとする要望に応えるための Unicode 文字
         ブロックの再分割。例えば v0.29 の改版では Halfwidth and Fullwidth
         Forms (ブロックID cjk12)が「再分割」されている。

    本パッケージでは、ブロックの分割の互換性について以下の方針を採る。
      - a のタイプの変更は「ある版の upTeX で未対応の文字ブロックの文字は
        その版では決して使われない」ことを仮定すれば互換性を損なうことが
        ない。従って、これに対しては特に対策を行わない。
      - b のタイプの変更(「再分割」と呼ぶ)は、既存のソースの動作を変更
       することになるので、これに対しては互換性の為の対策を行う。
      - ブロック ID を用いたブロック指定については、その対象文字集合が
        upTeX の版により変わらないようにする。
      - モード指定(\cjkcategorymode)を行う場合は、それによる kcatcode の
        設定値が upTeX の版により変わらないようにする。(ただし、その版に
        対応する Unicode の版で追加された文字を除く。)
      - パッケージを読み込むだけでは kcatcode への変更は一切行われない。
        従って、この場合の設定値は当然 upTeX の版に依存する。

    具体的な対応をこれから述べる。まず、文字ブロックの「再分割」の違いを
    「文字分類バージョン(CCV;Character Category Version)」と呼ぶことに
    する。現状では次のものが存在する。

        CCV 1 : upTeX v0.11 での定義
        CCV 2 : upTeX v0.29 での定義

    その上で、分割の変更について、以下のように対応する。ここでは CCV 2 に
    おける cjk12 の再分割を例にする。

      - cjk12 の再分割に関しては、再分割後のブロックに新たに ID を与える
        (cjk1a, cjk1b, cjk1c)。cjk12 も引き続き使用可能である。CCV が 2
        以降の upTeX で cjk12 の kcatcode を変更する場合には、内部では
        cjk1a~cjk1c を同時に変更する動作を行う。当然ながら CCV 1 の upTeX
        では cjk1a~cjk1c の指定は使えない。

    CCV 2 では分割が変更されただけでなく、cjk1b と cjk1c の kcatcode の
    既定値が 18(cjk) から 17(kana) に変更される。本パッケージでのモード
    設定における kcatcode の設定値は upTeX の既定値を基礎としていて、特に
    forcecjk は既定値と全く同じ設定としている。既定値の変更については以下
    のように対応する。

      - パッケージオプションにおいて、「モード設定の際の設定値の基礎とする
        CCV」(これを「モード CCV」と呼ぶ)を指定できるようにする。すなわち
        'ccvN' (N=1~2) でモード CCV が N になる。
      - forcecjk の設定値は CCV が「モード CCV」である upTeX の既定値と
        一致する。それ以外のモードの設定値もそれに応じて変わる。
      - モード CCV の既定値は 1 とする。従って、*モード設定を使用した場合*
        は、kcatcode の設定値は用いる upTeX の CCV に依存しない。
      - モード設定オプション無しでパッケージを読み込んだだけの場合は、
        kcatcode の設定は何も変更されない。従ってこの場合の設定値は用いる
        upTeX の CCV に依存する。

    例として次の場合を考える。
      [upTeX の CCV が 1]
        (1a) \usepackage{pxcjkcat}
        (1b) \usepackage[forcecjk]{pxcjkcat}
        (1c) \usepackage[ccv1,forcecjk]{pxcjkcat}
      [upTeX の CCV が 2]
        (2a) \usepackage{pxcjkcat}
        (2b) \usepackage[forcecjk]{pxcjkcat}
        (2c) \usepackage[ccv1,forcecjk]{pxcjkcat}
        (2d) \usepackage[ccv2,forcecjk]{pxcjkcat}
    (1a)(1b)(1c)(2b)(2c) は CCV 1 の既定値、(2a)(2d) は CCV 2 の既定値と
    同じ値に設定される。

   最後の注意点として、\cjkcategory でブロック指定に「文字」を使った場合
   は単純にその文字の属する文字ブロックとみなされ、その動作は用いる upTeX
   の CCV に依存し、モード CCV とは無関係である。

■ Unicode ブロック ID 一覧

    文字分類バージョン(CCV) 1 でのブロック。ただし [2] の注釈を付したもの
    は CCV 2 の upTeX で追加されたブロックである(先述の通り、Unicode 文字
    ブロック追加による変更に関しては CCV は関知しない)。CCV 1 の upTeX で
    は、例えば nkoo の範囲は実際には U+07C0~08FF となる。

    ID          Unicode 範囲   名称
    latn        0000 ..   007F Basic Latin
    latn1       0080 ..   00FF Latin-1 Supplement
    latnA       0100 ..   017F Latin Extended-A
    latnB       0180 ..   024F Latin Extended-B
    latn2       0250 ..   02AF IPA Extensions
    sym01       02B0 ..   02FF Spacing Modifier Letters
    sym02       0300 ..   036F Combining Diacritical Marks
    grek        0370 ..   03FF Greek and Coptic
    cyrl        0400 ..   04FF Cyrillic
    cyrl1       0500 ..   052F Cyrillic Supplement
    armn        0530 ..   058F Armenian
    hebr        0590 ..   05FF Hebrew
    arab        0600 ..   06FF Arabic
    syrc        0700 ..   074F Syriac
    arab1       0750 ..   077F Arabic Supplement
    thaa        0780 ..   07BF Thaana
    nkoo        07C0 ..   07FF NKo
    samr  [2]   0800 ..   08FF Samaritan
    deva        0900 ..   097F Devanagari
    beng        0980 ..   09FF Bengali
    guru        0A00 ..   0A7F Gurmukhi
    gujr        0A80 ..   0AFF Gujarati
    orya        0B00 ..   0B7F Oriya
    taml        0B80 ..   0BFF Tamil
    telu        0C00 ..   0C7F Telugu
    knda        0C80 ..   0CFF Kannada
    mlym        0D00 ..   0D7F Malayalam
    sinh        0D80 ..   0DFF Sinhala
    thai        0E00 ..   0E7F Thai
    laoo        0E80 ..   0EFF Lao
    tibt        0F00 ..   0FFF Tibetan
    mymr        1000 ..   109F Myanmar
    geor        10A0 ..   10FF Georgian
    hang1       1100 ..   11FF Hangul Jamo
    ethi        1200 ..   137F Ethiopic
    ethi1       1380 ..   139F Ethiopic Supplement
    cher        13A0 ..   13FF Cherokee
    cans        1400 ..   167F Unified Canadian Aboriginal Syllabics
    ogam        1680 ..   169F Ogham
    runr        16A0 ..   16FF Runic
    tglg        1700 ..   171F Tagalog
    hano        1720 ..   173F Hanunoo
    buhd        1740 ..   175F Buhid
    tagb        1760 ..   177F Tagbanwa
    khmr        1780 ..   17FF Khmer
    mong        1800 ..   187F Mongolian
    cans1 [2]   1880 ..   18FF Unified Canadian Aboriginal Syllabics
                                 Extended
    limb        1900 ..   194F Limbu
    tale        1950 ..   197F Tai Le
    talu        1980 ..   19DF New Tai Lue
    khmr1       19E0 ..   19FF Khmer Symbols
    bugi        1A00 ..   1A1F Buginese
    lana  [2]   1A20 ..   1AFF Tai Tham
    bali        1B00 ..   1B7F Balinese
    sund  [2]   1B80 ..   1BFF Sundanese
    lepc  [2]   1C00 ..   1C4F Lepcha
    olck  [2]   1C50 ..   1CCF Ol Chiki
    sym38 [2]   1CD0 ..   1CFF Vedic Extensions
    latn4       1D00 ..   1D7F Phonetic Extensions
    latn5       1D80 ..   1DBF Phonetic Extensions Supplement
    sym03       1DC0 ..   1DFF Combining Diacritical Marks Supplement
    latn3       1E00 ..   1EFF Latin Extended Additional
    grek1       1F00 ..   1FFF Greek Extended
    sym04       2000 ..   206F General Punctuation
    sym05       2070 ..   209F Superscripts and Subscripts
    sym06       20A0 ..   20CF Currency Symbols
    sym07       20D0 ..   20FF Combining Diacritical Marks for Symbols
    sym08       2100 ..   214F Letterlike Symbols
    sym09       2150 ..   218F Number Forms
    sym10       2190 ..   21FF Arrows
    sym11       2200 ..   22FF Mathematical Operators
    sym12       2300 ..   23FF Miscellaneous Technical
    sym13       2400 ..   243F Control Pictures
    sym14       2440 ..   245F Optical Character Recognition
    sym15       2460 ..   24FF Enclosed Alphanumerics
    sym16       2500 ..   257F Box Drawing
    sym17       2580 ..   259F Block Elements
    sym18       25A0 ..   25FF Geometric Shapes
    sym19       2600 ..   26FF Miscellaneous Symbols
    sym20       2700 ..   27BF Dingbats
    sym21       27C0 ..   27EF Miscellaneous Mathematical Symbols-A
    sym22       27F0 ..   27FF Supplemental Arrows-A
    brai        2800 ..   28FF Braille Patterns
    sym23       2900 ..   297F Supplemental Arrows-B
    sym24       2980 ..   29FF Miscellaneous Mathematical Symbols-B
    sym25       2A00 ..   2AFF Supplemental Mathematical Operators
    sym26       2B00 ..   2BFF Miscellaneous Symbols and Arrows
    glag        2C00 ..   2C5F Glagolitic
    latnC       2C60 ..   2C7F Latin Extended-C
    copt        2C80 ..   2CFF Coptic
    geor1       2D00 ..   2D2F Georgian Supplement
    tfng        2D30 ..   2D7F Tifinagh
    ethi2       2D80 ..   2DDF Ethiopic Extended
    cyrlA [2]   2DE0 ..   2DFF Cyrillic Extended-A
    sym27       2E00 ..   2E7F Supplemental Punctuation
    cjk01       2E80 ..   2EFF CJK Radicals Supplement
    cjk02       2F00 ..   2FEF Kangxi Radicals
    cjk03       2FF0 ..   2FFF Ideographic Description Characters
    cjk04       3000 ..   303F CJK Symbols and Punctuation
    hira        3040 ..   309F Hiragana
    kana        30A0 ..   30FF Katakana
    bopo        3100 ..   312F Bopomofo
    hang2       3130 ..   318F Hangul Compatibility Jamo
    cjk05       3190 ..   319F Kanbun
    bopo1       31A0 ..   31BF Bopomofo Extended
    cjk06       31C0 ..   31EF CJK Strokes
    kana1       31F0 ..   31FF Katakana Phonetic Extensions
    cjk07       3200 ..   32FF Enclosed CJK Letters and Months
    cjk08       3300 ..   33FF CJK Compatibility
    haniA       3400 ..   4DBF CJK Unified Ideographs Extension A
    sym28       4DC0 ..   4DFF Yijing Hexagram Symbols
    hani        4E00 ..   9FFF CJK Unified Ideographs
    yiii        A000 ..   A48F Yi Syllables
    yiii1       A490 ..   A4CF Yi Radicals
    lisu  [2]   A4D0 ..   A4FF Lisu
    vaii  [2]   A500 ..   A63F Vai
    cyrlB [2]   A640 ..   A69F Cyrillic Extended-B
    bamu  [2]   A6A0 ..   A6FF Bamum
    sym29       A700 ..   A71F Modifier Tone Letters
    latnD       A720 ..   A7FF Latin Extended-D
    sylo        A800 ..   A82F Syloti Nagri
    sym39 [2]   A830 ..   A83F Common Indic Number Forms
    phag        A840 ..   AB7F Phags-pa
    saur  [2]   A880 ..   A8DF Saurashtra
    deva1 [2]   A8E0 ..   A8FF Devanagari Extended
    kali  [2]   A900 ..   A92F Kayah Li
    rjng  [2]   A930 ..   A95F Rejang
    hangA [2]   A960 ..   A97F Hangul Jamo Extended-A
    java  [2]   A980 ..   A9FF Javanese
    cham  [2]   AA00 ..   AA5F Cham
    mymrA [2]   AA60 ..   AA7F Myanmar Extended-A
    tavt  [2]   AA80 ..   ABBF Tai Viet
    mtei  [2]   ABC0 ..   ABFF Meetei Mayek
    hang        AC00 ..   D7AF Hangul Syllables
    hangB [2]   D7B0 ..   D7FF Hangul Jamo Extended-B
    spc01       D800 ..   DB7F High Surrogates
    spc02       DB80 ..   DBFF High Private Use Surrogates
    spc03       DC00 ..   DFFF Low Surrogates
    spc04       E000 ..   F8FF Private Use Area
    hani1       F900 ..   FAFF CJK Compatibility Ideographs
    latn6       FB00 ..   FB4F Alphabetic Presentation Forms
    arab2       FB50 ..   FDFF Arabic Presentation Forms-A
    spc05       FE00 ..   FE0F Variation Selectors
    cjk09       FE10 ..   FE1F Vertical Forms
    sym30       FE20 ..   FE2F Combining Half Marks
    cjk10       FE30 ..   FE4F CJK Compatibility Forms
    cjk11       FE50 ..   FE6F Small Form Variants
    arab3       FE70 ..   FEFF Arabic Presentation Forms-B
    cjk12       FF00 ..   FFEF Halfwidth and Fullwidth Forms
    spc06       FFF0 ..   FFFF Specials                               
    linb       10000 ..  1007F Linear B Syllabary
    linb1      10080 ..  100FF Linear B Ideograms
    sym31      10100 ..  1013F Aegean Numbers
    grek2      10140 ..  1018F Ancient Greek Numbers
    sym40 [2]  10190 ..  101CF Ancient Symbols
    sym41 [2]  101D0 ..  1027F Phaistos Disc
    lyci  [2]  10280 ..  1029F Lycian
    cari  [2]  102A0 ..  102FF Carian
    ital       10300 ..  1032F Old Italic
    goth       10330 ..  1037F Gothic
    ugar       10380 ..  1039F Ugaritic
    xpeo       103A0 ..  103FF Old Persian
    dsrt       10400 ..  1044F Deseret
    shaw       10450 ..  1047F Shavian
    osma       10480 ..  107FF Osmanya
    cprt       10800 ..  1083F Cypriot Syllabary
    armi  [2]  10840 ..  108FF Imperial Aramaic
    phnx       10900 ..  1091F Phoenician
    lydi  [2]  10920 ..  109FF Lydian
    khar       10A00 ..  10A5F Kharoshthi
    sarb  [2]  10A60 ..  10AFF Old South Arabian
    avst  [2]  10B00 ..  10B3F Avestan
    prti  [2]  10B40 ..  10B5F Inscriptional Parthian
    phli  [2]  10B60 ..  10BFF Inscriptional Pahlavi
    orkh  [2]  10C00 ..  10E5F Old Turkic
    sym42 [2]  10E60 ..  1107F Rumi Numeral Symbols
    kthi  [2]  11080 ..  11FFF Kaithi
    xsux       12000 ..  123FF Cuneiform
    xsux1      12400 ..  12FFF Cuneiform Numbers and Punctuation
    egyp  [2]  13000 ..  1CFFF Egyptian Hieroglyphs
    sym32      1D000 ..  1D0FF Byzantine Musical Symbols
    sym33      1D100 ..  1D1FF Musical Symbols
    sym34      1D200 ..  1D2FF Ancient Greek Musical Notation
    sym35      1D300 ..  1D35F Tai Xuan Jing Symbols
    sym36      1D360 ..  1D3FF Counting Rod Numerals
    sym37      1D400 ..  1EFFF Mathematical Alphanumeric Symbols
    sym43 [2]  1F000 ..  1F02F Mahjong Tiles
    sym44 [2]  1F030 ..  1F0FF Domino Tiles
    sym45 [2]  1F100 ..  1F1FF Enclosed Alphanumeric Supplement
    cjk13 [2]  1F200 ..  1FFFF Enclosed Ideographic Supplement
    haniB      20000 ..  2A6FF CJK Unified Ideographs Extension B
    haniC [2]  2A700 ..  2F7FF CJK Unified Ideographs Extension C
    hani2      2F800 ..  2FFFF CJK Compatibility Ideographs Supplement
    spc07      E0000 ..  E00FF Tags
    spc08      E0100 ..  EFFFF Variation Selectors Supplement
    spc09      F0000 ..  FFFFF Supplementary Private Use Area-A
    spc10     100000 .. 10FFFF Supplementary Private Use Area-B

    ※ ID の命名規則
    - Unicode ブロック名にスクリプト(用字系)の名前が含まれるものは、それ
      に対する ISO 15924 のコードを用いた。単一のスクリプトのブロックが
      複数ある場合は、名前が "Extended-A, B, ..." のものは A, B, ... を、
      それ以外のものは 1, 2, ... (一部符号値順でない)を末尾に付加した。
        "Arabic" → arab ; "Latin Extended-C" → latnC
      なお、Hiragana は hira だが Katakana は kana であることに注意。
    - それ以外は、CJK 関係(cjk)、特殊用途(spc)、それ以外(sym)の 3 つに
      恣意的に分類して、2 桁の番号を付けた。この番号は基本的に符号値の
      順とするが、後から追加されたブロックはそうならない。

    CCV 2 では以下の変更がある。

    - cjk12 (Halfwidth and Fullwidth Forms) が次の 3 つに分割される。
      * cjk1b : U+FF10..U+FF19, U+FF21..U+FF3A, U+FF41..U+FF5A
        すなわち ASCII 英数字の全角互換形。
      * cjk1c : U+FF66..U+FF6F, U+FF71..U+FF9D
        すなわちカタカナの半角互換形。句読点等の記号は含まない。
      * cjk1a : cjk12 から cjk1b, cjk1c を除いた残り。
      cjk12 の指定も使用可能である。

■ 各モードでの kcatcode の値

    モード CCV が 1 の場合。

    記号    fc  pc pcv pnc JIS CID  名称
    latn    X   -   -   -       *   Basic Latin
    latn1   O   -   -   X   *   *   Latin-1 Supplement
    latnA   O   -   -   X       *   Latin Extended-A
    latnB   O   -   -   X       *   Latin Extended-B
    latn2   O   -   -   X       *   IPA Extensions
    sym01   O   -   -   X       *   Spacing Modifier Letters
    sym02   O   -   -   X       *   Combining Diacritical Marks
    grek    O   -   X   -   *   *   Greek and Coptic
    cyrl    O   -   X   -   *   *   Cyrillic
    hang1   H   -   -   -       +   Hangul Jamo
    latn3   O   -   -   X       *   Latin Extended Additional
    grek1   O   -   X   -       *   Greek Extended
    sym04   O   -   -   X   *   *   General Punctuation
    sym05   O   -   -   X       *   Superscripts and Subscripts
    sym06   O   -   -   X       *   Currency Symbols
    sym07   O   -   -   X       *   Combining Diacritical Marks for Symbols
    sym08   O   -   -   X   *   *   Letterlike Symbols
    sym09   O   -   -   X       *   Number Forms
    sym10   O   -   -   X   *   *   Arrows
    sym11   O   -   -   X   *   *   Mathematical Operators
    sym12   O   -   -   X   *   *   Miscellaneous Technical
    sym13   O   -   -   X       *   Control Pictures
    sym15   O   -   -   -       *   Enclosed Alphanumerics
    sym16   O   -   -   X   *   *   Box Drawing
    sym17   O   -   -   X       *   Block Elements
    sym18   O   -   -   X   *   *   Geometric Shapes
    sym19   O   -   -   X   *   *   Miscellaneous Symbols
    sym20   O   -   -   X       *   Dingbats
    sym23   O   -   -   X       *   Supplemental Arrows-B
    sym24   O   -   -   X       *   Miscellaneous Mathematical Symbols-B
    sym26   O   -   -   X       *   Miscellaneous Symbols and Arrows
    cjk01   I   -   -   -       *   CJK Radicals Supplement
    cjk02   I   -   -   -       *   Kangxi Radicals
    cjk03   I   -   -   -       +   Ideographic Description Characters
    cjk04   O   -   -   -   *   *   CJK Symbols and Punctuation
    hira    K   -   -   -   *   *   Hiragana
    kana    K   -   -   -   *   *   Katakana
    bopo    I   -   -   -       +   Bopomofo
    hang2   H   -   -   -       +   Hangul Compatibility Jamo
    cjk05   I   -   -   -       *   Kanbun
    bopo1   I   -   -   -       +   Bopomofo Extended
    cjk06   I   -   -   -       +   CJK Strokes
    kana1   K   -   -   -       *   Katakana Phonetic Extensions
    cjk07   O   -   -   -       *   Enclosed CJK Letters and Months
    cjk08   O   -   -   -       *   CJK Compatibility
    haniA   I   -   -   -       *   CJK Unified Ideographs Extension A
    hani    I   -   -   -   *   *   CJK Unified Ideographs
    yiii    O   -   -   -       +   Yi Syllables
    yiii1   O   -   -   -       +   Yi Radicals
    hangA   H   -   -   -           Hangul Jamo Extended-B
    hang    H   -   -   -       +   Hangul Syllables
    hangB   H   -   -   -           Hangul Jamo Extended-B
    spc04   O   -   -   X       *   Private Use Area
    hani1   I   -   -   -       *   CJK Compatibility Ideographs
    latn6   O   -   -   X       *   Alphabetic Presentation Forms
    cjk09   O   -   -   -       *   Vertical Forms
    cjk10   O   -   -   -       *   CJK Compatibility Forms
    cjk11   O   -   -   -       +   Small Form Variants
    cjk12   O   -   -   -   *   *   Halfwidth and Fullwidth Forms
    cjk13   O   -   -   -           Enclosed Ideographic Supplement
    haniB   I   -   -   -       *   CJK Unified Ideographs Extension B
    haniC   I   -   -   -           CJK Unified Ideographs Extension C
    hani2   I   -   -   -       *   CJK Compatibility Ideographs Supplement
            O   X   -   -           (その他のブロック)
    all, prc, pc1, pnc の欄はそれぞれ forcecjk, prefercjk, prefercjkvar,
    prefernoncjk モードでの kcatcode の値。値の意味: X=noncjk(15),
    I=kanji(16), K=kana(17), O=cjk(18), H=hangul(19)。JIS 欄の * はその
    ブロック内の文字に JIS X 0208 に含まれるものがあることを示す。また CID
    欄の * は Adobe-Japan1 に、+ は Adobe-GB1, Adobe-CNS1, Adobe-Korea1 に
    含まれるものがあることを示す。

    モード CCV が 2 の場合、cjk12 の再分割について以下のようになる。

    記号    fc  pc pcv pnc JIS CID
    cjk1a   O   -   -   -   *   *   cjk12 で cjk1b, cjk1c 以外
    cjk1b   K   -   -   -   *   *   ASCII 英数字の全角互換形
    cjk1c   K   -   -   -   *   *   カタカナの半角互換形(記号は含まず)

    ※ upTeX の CCV が 2 でモード CCV 指定が 1 の場合は cjk1a~cjk1c の
    設定値は cjk12 のものに従う。

----------------------------------------
\end{verbatim}
\end{document}
